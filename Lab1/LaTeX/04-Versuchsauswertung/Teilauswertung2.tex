% Teilauswertung X

\section{Questions}
\section{Questions}
The following questions had to be answered as part of the analysis:

\begin{itemize}
\item[a)] Does the correction of the apparent half solar diameter $\Phi_1$ has to be done before or after the correction of the atmospheric refraction?
\\ \\
The correction of the apparent half solar diameter $\Phi_1$ has to be done after the correction of the atmospheric refraction. The atmospheric refraction leads to a smaller zenith angle and it is altitude dependent. That is, the strength of atmospheric refraction depends on the amount of atmosphere the respective light has to passe. Therefore, we have to first take this error into account to then determine the position of the Sun. Once we have done that, we can use the apparent solar radius to determine the position of the upper or lower limb of the Sun.
\item[b)] The weather forecast always lists the air pressure reduced to sea level.  Are the refraction tables still valid for this value or the local pressure at the site of observation?
\\ \\
The values for the local pressure at the site of observation are valid as the values standardized to sea level underestimate the refraction error since the air density is lower at altitudes higher than the sea level. 
\item[c)] How accurate (estimate the magnitude) has the time to be measured,  if one wants to obtain $b$ with an error less than $1''$ and the observations are done at 12:00 and 15:00?
\item[d)] Which of the three solutions for $b$ is the correct one in your case? The possible combinations would result in 4 solutions.  What about the 4th solution?
\\ \\
Two of the solutions are the same, this is why we end up with three solution. Which solution is the correct one depends on whether it is summer or winter and whether the observer is located at the northern or southern hemisphere. As we observed in summer in the northern hemisphere, the correct solution for us is the ?? One. 
\item[e)] What would be the best time for your observations? Why? Is this method suitable for any possible observation time?
\\ \\ 
In principle, the best time would be if the Sun at the zenith as then the light this way has to pass the minimum amount of atmosphere, which would minimize the atmospheric refraction error. However, directly in the zenith it is hard to differentiate between the upper and lower limb of the Sun and also the contrasts are very low.
\item[f)]How can the longitude l be easily measured (by using a clock)?
\\ \\
Since the Earth rotates $360°$ in $24h$, we know that $15° = 1h$, $1° = 4min$, $1’ = 4s$
Thus, if we know the UTC at our location and our local solar noon, we can compute the longitude l of our location. 
E.g., if our local solar noon was at 12:00 UTC, we would know that we are at $l=0$° Greenwich. If, for example, out local solar noon was at 13:00 UTC, we would know that we are at $l=15°$ east of Greenwich and so on. 
\end{itemize}