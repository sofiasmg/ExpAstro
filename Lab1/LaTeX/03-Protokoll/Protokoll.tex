% 3. Protokoll

% Variables
\def\skalierung{0.65}

\chapter{Procedure}
\label{chap:protokoll}


% The theodolite is set up on a tripod and leveled. Once the theodolite is properly aligned, an exact horizontal orientation is performed, followed by an initial measurement of any alignment errors.

% The main observation begins by measuring the zenith angle at both the lower and upper edges of the solar disk. Each edge is measured at least ten times, and the exact time of each individual measurement is recorded, resulting in a total of twenty precise observations.

% After the solar measurements, the horizontal alignment is checked again to identify any changes or drifts during the procedure. The index error of the theodolite is then determined by measuring the zenith angle of a fixed terrestrial object, such as an antenna. This measurement is performed twice—once with the theodolite in its original position, and again after rotating it by 180°.

% To account for atmospheric effects, current weather conditions are noted, including temperature and air pressure. These values are recorded both before and after the observations, along with any unusual weather anomalies.

% Finally, the geographical latitude of the observation site is calculated using data from the Astronomical Almanac, incorporating individual refraction corrections for each data point. The height, latitude, and longitude of the site are also determined using a GPS device, with all relevant details documented for later analysis.



% Einbindung des Protokolls als pdf (mit Seitenzahl etc.)
% Erste Seite mit Überschrift
%\includepdf[pages = 1, landscape = false, nup = 1x1, scale = \skalierung , pagecommand={\thispagestyle{empty}\chapter{Protokoll}}]
%            {03-Protokoll/Protokoll.pdf}
% Restliche Seiten richtig skaliert
%\includepdf[pages = -, landscape = false, nup = 1x1, scale = \skalierung , pagecommand={}]
%            {03-Protokoll/Protokoll.pdf}